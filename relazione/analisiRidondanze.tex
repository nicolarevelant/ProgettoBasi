\section{Analisi ridondanze}
\label{ridondanze}

\subsection{Tabella operazioni}

% TODO: scegliere codici condominiali

\begin{tabular}{|p{320pt}|l|}
	\hline
	\textbf{Operazione} & \textbf{Frequenza} \\ \hline
	Modifica la quota dell'anno corrente dell'appartamento n° 3 del condominio "X" & 45 volte/mese \\ \hline
	Cancella condominio con codice "Y" & 0.2 volte/anno \\ \hline
	Inserimento Appartamento & 1 volta/anno \\ \hline
	Query ammontare complessivo per ogni condominio & 4 volte/anno \\ \hline
	Query indirizzo di tutti i proprietari & 1 volta/giorno \\ \hline
	Query dato x proprietario per ogni condominio avente almeno 1 app. posseduto da x, elencare le ultime 5 spese dal registro spese & 2 volte/mese \\ \hline
	Query elenco spese dell'anno corrente dei condomini che possiedono almeno 10 appartamenti & 1 volta/anno \\ \hline
	Query importo complessivo delle spese di tutti i condomini con $50 <= ammontareComplessivo <= 100$ & 5 volte/anno \\ \hline
	Query elenco persone che possiedono l'appartamento in cui abitano & 3 volte/anno \\ \hline
	Query elenco persone più anziane che possiedono un appartamento con $superficie >= 50$ & 2 volte/mese \\ \hline
\end{tabular}

\subsection{Tabella valori}

\begin{tabular}{|l|l|l|}
	\hline
	Concetto & Tipo & Volume \\ \hline
	Persona & Entità & 1000 \\ \hline
	Proprietario & Entità & 200 \\ \hline
	Appartamento & Entità & 1500 \\ \hline
	Condominio & Entità & 150 \\ \hline
	Spesa & Entità & 4500 \\ \hline
	abita & Relazione & 1000 \\ \hline
	possiede & Relazione & 1500 \\ \hline
	appartenenza & Relazione & 1500 \\ \hline
	paga & Relazione & 4500 \\ \hline
\end{tabular}

\subsection{Analisi ridondanza sull'attributo derivato Ammontare complessivo di Condominio}

L'analisi delle ridondanze è stata effettuata tenendo in considerazione l'attributo derivato \textbf{Ammontare complessivo} dell'entità \textbf{Condominio}, andando a calcolare il costo delle seguenti due operazioni nel caso in cui è presente l'attributo derivato oppure no:

\[
\begin{aligned}
\text{OP1} & := \text{inserimento Appartamento} \\
\text{OP2} & := \text{calcolare ammontare complessivo di un Condominio}
\end{aligned}
\]

Con frequenza rispettivamente di 1 volta/anno e 4 volte/anno

La seguente tabella ci sarà utile in seguito per calcolare il costo delle operazioni.

\bigskip

\begin{tabular}{|l|l|}
	\hline
	Operazione & Costo (u) \\ \hline
	Scrittura (w) & 2 \\ \hline
	Lettura (r) & 1 \\ \hline
\end{tabular}

\bigskip

L'obbiettivo che ci poniamo è quello di dimostrare che tenere l'attributo derivato sia computazionalmente vantaggioso, nel caso delle due operazioni in esame. Focalizziamo la nostra attenzione sulle entità \textbf{Condominio} e \textbf{Appartamento} e sulla relazione \textbf{Appartenenza}.

\subsubsection{Costo delle due operazione nel caso in cui la ridondanza venga tolta}

Per quanto riguarda l'operazione 1 abbiamo bisogno di un accesso in scrittura all'entità \textbf{Appartamento} e un accesso in scrittura alla relazione Appartenenza.

Per quanto riguarda l'operazione 2 serve un accesso in lettura all'entità \textbf{Condominio}, per ricavare il condominio in questione e 10 letture alla relazione \textbf{Appartenenza} (ottenuto dividendo il volume dell'entità \textbf{Appartamento} per il volume dell'entità \textbf{Condominio}).

Quindi,

\begin{samepage}
    \begin{align*}
        Costo_{OP1} &= 2w \\
        Costo_{OP2} &= 1r + \left(\frac{1500}{150}\right)r = 11r
    \end{align*}
\end{samepage}

Andando a moltiplicare i costi per le relative frequenze delle due operazioni e tenendo in considerazione la tabella subito sopra

\begin{samepage}
    \begin{align*}
        &Costo_{OP1} = 2 \cdot 2 \cdot 1 \text{ volta/anno} = 4 \text{ accessi/anno} \\
        &Costo_{OP2} = 11 \cdot 1 \cdot 4 \text{ volte/anno} = 44 \text{ accessi/anno} \\
        &CostoTOT_{senza\_rid} = 48 \text{ accessi/anno}
    \end{align*}
\end{samepage}

\subsubsection{Costo delle due operazione nel caso in cui la ridondanza venga mantenuta}

Per quanto riguarda l'operazione 1 abbiamo bisogno di un accesso in scrittura all'entità \textbf{Appartamento} (per inserire l'appartamento), un accesso in scrittura alla relazione \textbf{Appartenenza} (per memorizzare la coppia condominio-appartamento), un accesso in lettura all'entità Condominio (per cercare il condominio in questione) e un accesso in scrittura all'entità Condominio (sommando all'attributo derivato il valore dell'attributo Quota-anno-corrente dell'appartamento appena inserito).

Per quanto riguarda l'operazione 2 serve un solo accesso in lettura all'entità \textbf{Condominio}, per leggere il contenuto dell'attributo derivato Ammontare-complessivo.

Quindi,

\begin{samepage}
	\begin{align*}
		Costo_{OP1} &= 1r + 3w \\
		Costo_{OP2} &= 1r
    \end{align*}
\end{samepage}

Andando a moltiplicare i costi per le relative frequenze delle due operazioni e tenendo in considerazione la tabella subito sopra

\begin{samepage}
    \begin{align*}
        &Costo_{OP1} = 1 + (3 \cdot 2) \cdot 1 \text{ volta/anno} = 7 \text{ accessi/anno} \\
        &Costo_{OP2} = 1 \cdot 4 \text{ volte/anno} = 4 \text{ accessi/anno} \\
        &CostoTOT_{con\_rid} = 11 \text{ accessi/anno}
    \end{align*}
\end{samepage}

E quindi siccome \( CostoTOT_{con\_rid} < CostoTOT_{senza\_rid} \), conviene mantenere l'attributo derivato Ammontare-complessivo.
