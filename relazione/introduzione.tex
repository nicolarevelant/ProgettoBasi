\section{Introduzione}

Questo progetto permette di gestire la base di dati di
un sistema condominiale composto da diversi condomini,
tenendo traccia delle persone in cui ci abitano, i
proprietari di ogni appartamento, la quota d'affitto
pagabile a rate e le spese condominiali.

\subsection{Descrizione soluzione}

Per implementare il sistema si parte dalla creazione di uno
\hyperref[schemaER]{schema concettuale}
di tipo ER\footnote{Entità/relazioni},
il quale definisce quali entità sono presenti nel problema
e come sono collegate fra di loro.

Lo schema concettuale viene poi analizzato alla ricerca di
eventuali
\hyperref[ridondanze]{attributi ridondanti}
per stabilire se conviene o meno mantenerli alla fase successiva.

La fase successiva è la
\hyperref[logico]{progettazione logica}
che modella il problema da un punto di vista legato al tipo di
DBMS\footnote{Database Management System - Sistema di gestione dei database}.

In questo caso si utilizza il modello logico relazionale che
utilizza le relazioni e le associazioni fra di esse. Tale schema
non usa le specializzazioni che vengono quindi eliminate.

% TODO: sistemare questa parte, non rispecchia la distinzione tra prog.fisica e l'implementazione...

Infine lo schema viene tradotto in linguaggio SQL per la
creazione del database e delle tabelle (o relazioni) le quali
verranno popolate con i dati generati da uno strumento esterno.

Tale strumento esterno è uno script in linguaggio R che si
occupa di creare le istruzioni di inserimento nella base di dati
e la generazione di grafici.

\subsection{Obiettivi della soluzione}

\begin{itemize}
    \item \textbf{Raccolta e analisi dei requisiti}: dopo aver letto attentamente i requisiti fornitoci abbiamo identificato i principali attori del problema e stilato un glossario dei termini, utile per avere una visione schematizzata del problema affrontato. In questa fase abbiamo fatto le opportune scelte per appianare eventuali dubbi e ambiguità presenti nel testo, anche confrontandoci con il professore.
    \item \textbf{Progettazione dello schema-ER}: tramite un approccio \textit{inside-out} abbiamo proceduto alla rappresentazione del nostro problema mediante uno schema ER, dove sono state definite le entità (Condominio, Spesa, Appartamento, Persona e Proprietario) e le relazioni (paga, appartenenza, possiede, abita). In questa fase sono stati definiti anche opportuni vincoli d'integrità che saranno poi gestiti tramite dei \textit{trigger} nella fase fisica.
    \item \textbf{Analisi delle ridondanze}: valuta, dopo opportuni calcoli, se mantenere o meno certi attributi ridondanti come l'ammontare complessivo del condominio per migliorare eventualmente l'efficienza generale del sistema.
    \item \textbf{Progettazione logica e fisica}: si traduce lo schema ER in un modello logico relazionale, si definiscono le chiavi esterne, le query e opportuni indici per migliorare l'efficienza del sistema.
    \item \textbf{Implementazione}: si definiscono le tabelle della base di dati, le query, gli indici e i trigger in \textit{SQL}, si popola la base di dati (tramite \textit{R}).
    \item \textbf{Analisi dei dati}: si comprendono i dati, la relazione tra essi, eventuali trend o pattern.
\end{itemize}
