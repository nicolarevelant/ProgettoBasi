\section{Introduzione}

Questo progetto permette di gestire la base di dati di
un sistema condominiale composto da diversi condomini,
tenendo traccia delle persone in cui ci abitano, i
proprietari di ogni appartamento, la quota d'affitto
pagabile a rate e le spese condominiali.

\subsection{Descrizione soluzione}

Per implementare il sistema si parte dalla creazione di uno
\hyperref[schemaER]{schema concettuale} di tipo ER\footnote{Entità/relazioni},
il quale definisce quali entità sono presenti nel problema
e come sono collegate fra di loro.

Poi viene costruito lo schema logico di tipo relazionale,
che è una traduzione dello schema concettuale in uno schema
interpretabile da un gestore di database relazionali.

Infine lo schema viene tradotto in linguaggio SQL per la
creazione del database e delle tabelle (o relazioni) le quali
verranno popolate con i dati generati da uno strumento esterno.

Tale strumento esterno è uno script in linguaggio R che si
occupa di creare le istruzioni di inserimento nella base di dati
e la generazione di grafici.
