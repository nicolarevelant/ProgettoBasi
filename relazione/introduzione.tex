\section{Introduzione}

Questo progetto permette di gestire la base di dati di
un sistema condominiale composto da diversi condomini,
tenendo traccia delle persone in cui ci abitano, i
proprietari di ogni appartamento, la quota d'affitto
pagabile a rate e le spese condominiali.

\subsection{Descrizione soluzione}

Per implementare il sistema si parte dalla creazione di uno
\hyperref[schemaER]{schema concettuale}
di tipo ER\footnote{Entità/relazioni},
il quale definisce quali entità sono presenti nel problema
e come sono collegate fra di loro.

Lo schema concettuale viene poi analizzato alla ricerca di
eventuali
\hyperref[ridondanze]{attributi ridondanti}
per stabilire se conviene o meno mantenerli alla fase successiva.

La fase successiva è la
\hyperref[logico]{progettazione logica}
che modella il problema da un punto di vista legato al tipo di
DBMS\footnote{Database Management System - Sistema di gestione dei database}.

In questo caso si utilizza il modello logico relazionale che
utilizza le relazioni e le associazioni fra di esse. Tale schema
non usa le specializzazioni che vengono quindi eliminate.

% TODO: sistemare questa parte, non rispecchia la distinzione tra prog.fisica e l'implementazione...

Infine lo schema viene tradotto in linguaggio SQL per la
creazione del database e delle tabelle (o relazioni) le quali
verranno popolate con i dati generati da uno strumento esterno.

Tale strumento esterno è uno script in linguaggio R che si
occupa di creare le istruzioni di inserimento nella base di dati
e la generazione di grafici.
