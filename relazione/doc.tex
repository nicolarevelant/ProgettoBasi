% Options for packages loaded elsewhere
\PassOptionsToPackage{unicode}{hyperref}
\PassOptionsToPackage{hyphens}{url}
%
\documentclass[
]{article}
\usepackage{amsmath,amssymb}
\usepackage{iftex}
\ifPDFTeX
  \usepackage[T1]{fontenc}
  \usepackage[utf8]{inputenc}
  \usepackage{textcomp} % provide euro and other symbols
\else % if luatex or xetex
  \usepackage{unicode-math} % this also loads fontspec
  \defaultfontfeatures{Scale=MatchLowercase}
  \defaultfontfeatures[\rmfamily]{Ligatures=TeX,Scale=1}
\fi
\usepackage{lmodern}
\ifPDFTeX\else
  % xetex/luatex font selection
\fi
% Use upquote if available, for straight quotes in verbatim environments
\IfFileExists{upquote.sty}{\usepackage{upquote}}{}
\IfFileExists{microtype.sty}{% use microtype if available
  \usepackage[]{microtype}
  \UseMicrotypeSet[protrusion]{basicmath} % disable protrusion for tt fonts
}{}
\makeatletter
\@ifundefined{KOMAClassName}{% if non-KOMA class
  \IfFileExists{parskip.sty}{%
    \usepackage{parskip}
  }{% else
    \setlength{\parindent}{0pt}
    \setlength{\parskip}{6pt plus 2pt minus 1pt}}
}{% if KOMA class
  \KOMAoptions{parskip=half}}
\makeatother
\usepackage{xcolor}
\setlength{\emergencystretch}{3em} % prevent overfull lines
\providecommand{\tightlist}{%
  \setlength{\itemsep}{0pt}\setlength{\parskip}{0pt}}
\setcounter{secnumdepth}{-\maxdimen} % remove section numbering
\ifLuaTeX
  \usepackage{selnolig}  % disable illegal ligatures
\fi
\IfFileExists{bookmark.sty}{\usepackage{bookmark}}{\usepackage{hyperref}}
\IfFileExists{xurl.sty}{\usepackage{xurl}}{} % add URL line breaks if available
\urlstyle{same}
\hypersetup{
  hidelinks,
  pdfcreator={LaTeX via pandoc}}

\author{}
\date{}

\begin{document}

\hypertarget{autori}{%
\section{Autori}\label{autori}}

\begin{itemize}
\tightlist
\item
  Nomi autori TODO
\end{itemize}

\hypertarget{introduzione}{%
\section{Introduzione}\label{introduzione}}

TODO: piccola descrizione del progetto

\hypertarget{descrizione-soluzione}{%
\section{Descrizione soluzione}\label{descrizione-soluzione}}

TODO: approccio usato (top down,\ldots)

\hypertarget{schema-entituxe0relazioni-er}{%
\section{Schema entità/relazioni
(ER)}\label{schema-entituxe0relazioni-er}}

\hypertarget{analisi-ridondanze}{%
\section{Analisi ridondanze}\label{analisi-ridondanze}}

\hypertarget{tabella-operazioni}{%
\subsection{Tabella operazioni}\label{tabella-operazioni}}

\hypertarget{tabella-valori}{%
\subsection{Tabella valori}\label{tabella-valori}}

\hypertarget{schema-logico-relazionale}{%
\section{Schema logico relazionale}\label{schema-logico-relazionale}}

Lo schema logico permette di rappresentare i concetti derivanti dallo
schema ER nel modello logico utilizzato dalla base di dati.

In questo progetto viene utilizzato il modello relazionale il quale
utilizza le relazioni (o tabelle) e le associazioni fra di esse per
rappresentare i dati richiesti dal modello concettuale.

Il seguente schema logico ha tradotto le entità dello schema ER in
tabelle, e le relazioni di tipo 1 a N dall'entità A all'entità B in
associazioni tra la chiave esterna di A che fa riferimento alla chiave
primaria di B.

In questo schema ER è presente una singola specializzazione parziale di
Persona in Proprietario pertanto viene unita al genitore, e tutti gli
attributi e relazioni del figlio ora sono sono del genitore.

L'attributo condominio.ammontareComplessivo è un attributo derivato ma è
comunque presente nello schema logico in quanto lo studio sulla
ridondanza ha sottolineato che mantenerlo porta una maggiore efficienza
computazionale della basi di dati.

\begin{quote}
condominio(\textbf{codice}, contoCorrente, indirizzo,
ammontareComplessivo)

spesa(\textbf{dataOra,} \textbf{\emph{condominio}}, importo, causale)

appartamento(\textbf{numero,} \textbf{\emph{condominio}},
quotaAnnoCorrente, sommaPagata, telefono, superficie,
\emph{proprietario})

persona(\textbf{cf}, nome, dataNascita, indirizzo,
\emph{numeroAppartamento, condominio})
\end{quote}

\hypertarget{progettazione-fisica}{%
\section{Progettazione fisica}\label{progettazione-fisica}}

\hypertarget{implementazione-in-sql}{%
\section{Implementazione in SQL}\label{implementazione-in-sql}}

\hypertarget{analisi-dati}{%
\section{Analisi dati}\label{analisi-dati}}

\end{document}
