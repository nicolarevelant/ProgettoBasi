\section{Analisi ridondanze}

\subsection{Tabella operazioni}

% TODO: scegliere codici condominiali

\begin{tabular}{|p{300pt}|l|}
	\hline
	\textbf{Operazione} & \textbf{Frequenza} \\ \hline
	Modifica la quota dell'anno corrente dell'appartamento n° 3 del condominio "X" & 45 volte/mese \\ \hline
	Cancella condominio con codice "Y" & 0.2 volte/anno \\ \hline
	Inserimento Appartamento & 1 volta/anno \\ \hline
	Query ammontare complessivo di tutti i condomini (calcolarlo) & 4 volte/anno \\ \hline
	Query indirizzo di tutti i proprietari & 1 volta/giorno \\ \hline
	Query dato x proprietario per ogni condominio avente almeno 1 app. posseduto da x, elencare le ultime 5 spese dal registro spese & 2 volte/mese \\ \hline
	Query elenco spese dell'anno corrente dei condomini che possiedono almeno 10 appartamenti & 1 volta/anno \\ \hline
	Query importo complessivo delle spese di tutti i condomini con $50 <= ammontareComplessivo <= 100$ & 5 volte/anno \\ \hline
	Query elenco persone che possiedono l'appartamento in cui abitano & 3 volte/anno \\ \hline
	Query elenco persone più anziane che possiedono un appartamento con $superficie >= 50$ & 2 volte/mese \\ \hline
\end{tabular}

\subsection{Tabella valori}

\begin{tabular}{|l|l|l|}
	\hline
	Concetto & Tipo & Volume \\ \hline
	Persona & Entità & 1000 \\ \hline
	Proprietario & Entità & 200 \\ \hline
	Appartamento & Entità & 1500 \\ \hline
	Condominio & Entità & 150 \\ \hline
	Spesa & Entità & 4500 \\ \hline
	abita & Relazione & 1000 \\ \hline
	possiede & Relazione & 1500 \\ \hline
	appartenenza & Relazione & 1500 \\ \hline
	paga & Relazione & 4500 \\ \hline
\end{tabular}

\subsection{Analisi ridondanza sull'attributo derivato Ammontare complessivo di Condominio}

L'analisi delle ridondanze è stata effettuata tenendo in considerazione l'attributo derivato Ammontare-Complessivo dell'entità Condominio, andando a calcolare il costo delle seguenti due operazioni nel caso in cui è presente l'attributo derivato oppure no:

\begin{samepage}
	
\begin{itemize}
	
	\item OP1 := inserimento Appartamento
	\item OP2 := calcolare ammontare complessivo di un Condominio

\end{itemize}

\end{samepage}

Con frequenza rispettivamente di 1 volta/anno e 4 volte/anno

La seguente tabella ci sarà utile in seguito per calcolare il costo delle operazioni.
| Operazione    | Costo (u) |
|---------------|-----------|
| Scrittura (w) | 2         |
| Lettura (r)   | 1         |

L'obbiettivo che ci poniamo è quello di dimostrare che tenere l'attributo derivato sia computazionalmente vantaggioso, nel caso delle due operazioni in esame. Focalizziamo la nostra attenzione sulle entità **Condominio** e **Appartamento** e sulla relazione **Appartenenza**.

***Costo delle due operazione nel caso in cui la ridondanza venga tolta***

Per quanto riguarda l'operazione 1 abbiamo bisogno di un accesso in scrittura all'entità Appartamento e un accesso in scrittura alla relazione Appartenenza.

Per quanto riguarda l'operazione 2 serve un accesso in lettura all'entità Condominio, per ricavare il condominio in questione e 10 letture alla relazione Appartenenza (ottenuto dividendo il volume dell'entità Appartamento per il volume dell'entità Condominio).

Quindi,

\begin{verbatim}
	
> Costo_OP1 = 2w
> 
> Costo_OP2 = 1r + (1500/150)r = 11r

\end{verbatim}

Andando a moltiplicare i costi per le relative frequenze delle due operazioni e tenendo in considerazione la tabella subito sopra

\begin{verbatim}
	
> Costo_OP1 = 2 * 2 * 1 volta/anno = 4 accessi all'anno
> 
> Costo_OP2 = 11 * 1 * 4 volte/anno = 44 accessi all'anno
>
> Costo_TOT_senza_rid = 48 accessi all'anno

\end{verbatim}

***Costo delle due operazione nel caso in cui la ridondanza venga mantenuta***

Per quanto riguarda l'operazione 1 abbiamo bisogno di un accesso in scrittura all'entità Appartamento (per inserire l'appartamento), un accesso in scrittura alla relazione Appartenenza (per memorizzare la coppia condominio-appartamento), un accesso in lettura all'entità Condominio (per cercare il condominio in questione) e un accesso in scrittura all'entità Condominio (sommando all'attributo derivato il valore dell'attributo Quota-anno-corrente dell'appartamento appena inserito).

Per quanto riguarda l'operazione 2 serve un solo accesso in lettura all'entità Condominio, per leggere il contenuto dell'attributo derivato Ammontare-complessivo.

Quindi,

\begin{verbatim}

> Costo_OP1 = 1r + 3w
> 
> Costo_OP2 = 1r

\end{verbatim}

Andando a moltiplicare i costi per le relative frequenze delle due operazioni e tenendo in considerazione la tabella subito sopra

\begin{verbatim}
	
> Costo_OP1 = 1 + (3 * 2) * 1 volta/anno = 7 accessi all'anno
> 
> Costo_OP2 = 1 * 4 volte/anno = 4 accessi all'anno
>
> Costo_TOT_con_rid = 11 accessi all'anno

\end{verbatim}

E quindi siccome $Costo\_TOT\_con\_rid < Costo\_TOT\_senza\_rid$ allora conviene mantenere l'attributo derivato Ammontare-complessivo.