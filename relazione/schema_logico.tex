\section{Schema logico relazionale}

Lo schema logico permette di rappresentare i concetti derivanti dallo schema ER
nel modello logico utilizzato dalla base di dati.

In questo progetto viene utilizzato il modello relazionale il quale utilizza le relazioni
(o tabelle) e le associazioni fra di esse per rappresentare i dati richiesti dal modello
concettuale.

Il seguente schema logico ha tradotto le entità dello schema ER in tabelle, e le relazioni
di tipo 1 a N dall'entità A all'entità B in associazioni tra la chiave esterna di A che
fa riferimento alla chiave primaria di B.

In questo schema ER è presente una singola specializzazione parziale di Persona in
Proprietario pertanto viene unita al genitore, e tutti gli attributi e relazioni del figlio
ora sono sono del genitore.

L'attributo condominio.ammontareComplessivo è un attributo derivato ma è comunque presente
nello schema logico in quanto lo studio sulla ridondanza ha sottolineato che mantenerlo porta
una maggiore efficienza computazionale della basi di dati.

\begin{itemize}

\item condominio(\underline{codice}, contoCorrente, indirizzo, ammontareComplessivo)

\item spesa(\underline{dataOra, \textit{condominio}}, importo, causale)

\item appartamento(\underline{numero, \textit{condominio}}, quotaAnnoCorrente, sommaPagata, telefono, superficie, \textit{proprietario})

\item persona(\underline{cf}, nome, dataNascita, indirizzo, \textit{numeroAppartamento, condominio})

\end{itemize}

\subsection{Chiavi esterne}

Di seguito sono elencate le chiavi esterne, la freccia indica che l'attributo (o l'insieme di attributi)
a sinistra è chiave esterna dell'entità a destra

% TODO: usare o no le parentesi graffe

\begin{itemize}
	\item spesa.condominio $\implies$ condominio
	\item appartamento.condominio $\implies$ condominio
	\item appartamento.proprietario $\implies$ persona
	\item $\{$persona.numeroAppartamento, persona.condominio$\}$ $\implies$ appartamento
\end{itemize}

