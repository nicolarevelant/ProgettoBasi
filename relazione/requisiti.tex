\section{Raccolta e analisi dei requisiti}
\label{requisiti}

\subsection{Glossario dei termini}

La fase di raccolta dei requisiti è una fase preliminare e  cruciale per garantire che il sistema progettato risponda adeguatamente alle necessità degli utenti finali. Questa fase coinvolge diverse attività, tra cui interviste con gli amministratori di condominio, analisi della documentazione esistente, e studio delle normative vigenti. Abbiamo ristrutturato il documento informale datoci tramite il seguente glossario dei termini:

\begin{table}[htbp]
    \centering
    \begin{adjustbox}{max width=\textwidth}
            \begin{tabular}{|l|l|l|}
                \hline
                \textbf{Termine} & \textbf{Descrizione} & \textbf{Collegamenti} \\ \hline
                Condominio & \begin{tabular}[c]{@{}l@{}}Edificio residenziale, amministrato da un singolo ente, che contiene\\ più appartamenti\end{tabular} & Spesa, Appartamento \\ \hline
                
                Appartamento & \begin{tabular}[c]{@{}l@{}}Singola unità abitativa, all'interno di un condominio, posseduta da\\ una sola persona \end{tabular}& Condominio, Persona \\ \hline
                
                Spesa & \begin{tabular}[c]{@{}l@{}}Costo che i proprietari di un appartamento sono tenuti a pagare\\ per coprire le spese comuni relative alla gestione del condominio\end{tabular} & Condominio \\ \hline
                
                Persona & \begin{tabular}[c]{@{}l@{}}Persona che possiede zero o più appartamenti, anche in diversi\\ condomini, e abita in un appartamento \end{tabular}& Appartamento \\ \hline
            \end{tabular}
    \end{adjustbox}
    
\end{table}

Per procedere senza particolari complicazioni d'ora in poi assumiamo che le persone possiedono (o abitano in) solo appartamenti dei condomini che sono gestiti dall'amministrazione.
