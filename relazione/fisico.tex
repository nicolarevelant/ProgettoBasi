\section{Progettazione fisica}
\label{fisico}

\subsection{Indici}

Tenendo in considerazione le seguenti query:

\begin{tabular}{|p{320pt}|l|}
	\hline
	\textbf{Operazione} & \textbf{Frequenza} \\ \hline
	Query elenco spese dell'anno corrente dei condomini che possiedono almeno 10 appartamenti & 1 volta/anno \\ \hline
	Query importo complessivo delle spese di tutti i condomini con $50 <= ammontareComplessivo <= 100$ & 5 volte/anno \\ \hline
	Query elenco persone più anziane che possiedono un appartamento con $superficie >= 50$ & 2 volte/mese \\ \hline
\end{tabular}

abbiamo optato per introdurre indici sui seguenti attributi:

\begin{itemize}
    \item proprietario della relazione Appartamento;
    \item ammontareComplessivo della relazione Condominio;
    \item dataNascita della relazione Persona;
    \item superficie della relazione Appartamento.
\end{itemize}

Il motivo per cui abbiamo utilizzato degli indici è per migliorare le prestazioni delle query sopracitate, perché o sono operazioni effettuate spesso (come nel caso dell'ultima query), o sono operazioni computazionalmente pesanti (come nel caso delle prime due query).
