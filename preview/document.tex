% Options for packages loaded elsewhere
\PassOptionsToPackage{unicode}{hyperref}
\PassOptionsToPackage{hyphens}{url}
%
\documentclass[
]{article}
\usepackage{amsmath,amssymb}
\usepackage{iftex}
\ifPDFTeX
  \usepackage[T1]{fontenc}
  \usepackage[utf8]{inputenc}
  \usepackage{textcomp} % provide euro and other symbols
\else % if luatex or xetex
  \usepackage{unicode-math} % this also loads fontspec
  \defaultfontfeatures{Scale=MatchLowercase}
  \defaultfontfeatures[\rmfamily]{Ligatures=TeX,Scale=1}
\fi
\usepackage{lmodern}
\ifPDFTeX\else
  % xetex/luatex font selection
\fi
% Use upquote if available, for straight quotes in verbatim environments
\IfFileExists{upquote.sty}{\usepackage{upquote}}{}
\IfFileExists{microtype.sty}{% use microtype if available
  \usepackage[]{microtype}
  \UseMicrotypeSet[protrusion]{basicmath} % disable protrusion for tt fonts
}{}
\makeatletter
\@ifundefined{KOMAClassName}{% if non-KOMA class
  \IfFileExists{parskip.sty}{%
    \usepackage{parskip}
  }{% else
    \setlength{\parindent}{0pt}
    \setlength{\parskip}{6pt plus 2pt minus 1pt}}
}{% if KOMA class
  \KOMAoptions{parskip=half}}
\makeatother
\usepackage{xcolor}
\usepackage{longtable,booktabs,array}
\usepackage{calc} % for calculating minipage widths
% Correct order of tables after \paragraph or \subparagraph
\usepackage{etoolbox}
\makeatletter
\patchcmd\longtable{\par}{\if@noskipsec\mbox{}\fi\par}{}{}
\makeatother
% Allow footnotes in longtable head/foot
\IfFileExists{footnotehyper.sty}{\usepackage{footnotehyper}}{\usepackage{footnote}}
\makesavenoteenv{longtable}
\setlength{\emergencystretch}{3em} % prevent overfull lines
\providecommand{\tightlist}{%
  \setlength{\itemsep}{0pt}\setlength{\parskip}{0pt}}
\setcounter{secnumdepth}{-\maxdimen} % remove section numbering
\ifLuaTeX
  \usepackage{selnolig}  % disable illegal ligatures
\fi
\IfFileExists{bookmark.sty}{\usepackage{bookmark}}{\usepackage{hyperref}}
\IfFileExists{xurl.sty}{\usepackage{xurl}}{} % add URL line breaks if available
\urlstyle{same}
\hypersetup{
  hidelinks,
  pdfcreator={LaTeX via pandoc}}

\author{}
\date{}

\begin{document}

\hypertarget{progetto-basi}{%
\section{Progetto Basi}\label{progetto-basi}}

\hypertarget{descrizione-attributi-derivati}{%
\subsection{Descrizione attributi
derivati}\label{descrizione-attributi-derivati}}

\begin{itemize}
\tightlist
\item
  Condominio.Ammontare-complessivo: sommatoria di ``quota anno
  corrente'' di ogni appartamento appartenente al condominio
\item
  Proprietario.indirizzo: Indirizzo del condominio a cui appartiene
  l'appartamento in cui vive
\item
  Appartamento.è-affittato: Vero sse ci abita una Persona che non è
  proprietario di tale appartamento
\end{itemize}

\hypertarget{vincoli-di-integrituxe0}{%
\subsection{Vincoli di integrità}\label{vincoli-di-integrituxe0}}

\begin{itemize}
\tightlist
\item
  Proprietario.indirizzo: NULL sse il proprietario abita in un
  appartamento che possiede
\item
  Ogni proprietario abita in un appartamento che possiede oppure paga
  l'affitto per abitare in un altro appartamento come inquilino
\end{itemize}

\hypertarget{tabella-operazioni}{%
\subsubsection{Tabella operazioni}\label{tabella-operazioni}}

\begin{longtable}[]{@{}
  >{\raggedright\arraybackslash}p{(\columnwidth - 2\tabcolsep) * \real{0.6786}}
  >{\raggedright\arraybackslash}p{(\columnwidth - 2\tabcolsep) * \real{0.3214}}@{}}
\toprule\noalign{}
\begin{minipage}[b]{\linewidth}\raggedright
Operazione
\end{minipage} & \begin{minipage}[b]{\linewidth}\raggedright
Frequenza
\end{minipage} \\
\midrule\noalign{}
\endhead
\bottomrule\noalign{}
\endlastfoot
Inserimento Proprietario & 2 volte/anno \\
Modifica Appartamento.Quota-anno-corrente & 45 volte/mese \\
Query Condominio.Ammontare-complessivo (calcolarlo) & 4 volte/anno \\
Cancella Condominio & 0.2 volte/anno \\
Query Proprietario.indirizzo & 1 volta/giorno \\
Query dato x proprietario per ogni condominio avente almeno 1 app.
posseduto da x, elencare le ultime 5 spese dal registro spese & 2
volte/mese \\
\end{longtable}

\hypertarget{tabella-valori}{%
\subsection{Tabella valori}\label{tabella-valori}}

\begin{longtable}[]{@{}lll@{}}
\toprule\noalign{}
Concetto & Tipo & Volume \\
\midrule\noalign{}
\endhead
\bottomrule\noalign{}
\endlastfoot
Persona & Entità & 1000 \\
Proprietario & Entità & 200 \\
Appartamento & Entità & 1500 \\
Condominio & Entità & 150 \\
Spesa & Entità & 4500 \\
abita & Relazione & 1000 \\
possiede & Relazione & 1500 \\
appartenenza & Relazione & 1500 \\
paga & Relazione & 4500 \\
\end{longtable}

\hypertarget{analisi-ridondanze}{%
\subsection{Analisi ridondanze}\label{analisi-ridondanze}}

\hypertarget{schema-logico}{%
\subsection{Schema logico}\label{schema-logico}}

L'attributo condominio.ammontareComplessivo è un attributo derivato ma è
comunque presente nello schema logico in quanto lo studio sulla
ridondanza ha sottolineato che manatenerlo porta una maggiore efficienza
computazionale.

Nello schema ER l'entità Proprietario è una specializzazione totale
dell'entità Persona. Dato che l'entità Persona non ha altre
specializzazioni, le 2 entità vengolo collise in 1 sola tabella
(persona)

\begin{quote}
condominio(\textbf{codice}, indirizzo, contoCorrente, indirizzo,
ammontareComplessivo)

spesa(\textbf{dataOra,} \textbf{\emph{condominio}}, importo, causale)

appartamento(\textbf{numero,} \textbf{\emph{condominio}},
quotaAnnoCorrente, sommaPagata, telefono, superficie,
\emph{proprietario})

persona(\textbf{cf}, nome, dataNascita, \emph{numeroAppartamento,
condominio})
\end{quote}

\hypertarget{chiavi-esterne}{%
\subsubsection{Chiavi esterne}\label{chiavi-esterne}}

Di seguito sono elencate le chiavi esterne, la freccia indica che
l'attributo a sinistra è chiave esterna dell'entità a destra

\begin{quote}
spesa.condominio --\textgreater{} condominio

appartamento.condominio --\textgreater{} condominio

appartamento.proprietario --\textgreater{} persona

\{persona.numeroAppartamento, persona.condominio\} --\textgreater{}
appartamento
\end{quote}

\hypertarget{note}{%
\section{Note}\label{note}}

\begin{itemize}
\tightlist
\item
  Rimozione età persona
\item
  Modifica delle frequenze della Tabella delle operazioni (tenendo conto
  della frequenza totale)
\end{itemize}

\end{document}
